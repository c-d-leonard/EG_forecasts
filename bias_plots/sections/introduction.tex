Modern cosmology hinges on observations of the large-scale structure of the Universe, which is rich with clues about gravity, dark energy, and the mechanisms of cosmic expansion. Next-generation galaxy surveys, including spectroscopic experiments such as the Dark Energy Spectroscopic Instrument (DESI, \citealt{DESI16}) and deep imaging experiments such as the Large Synoptic Survey Telescope (LSST, \citealt{LSST09}), will map billions of galaxies in the coming decade and tighten constraints on key fundamental parameters. While spectroscopic redshifts can be obtained for some subset of imaged galaxies, the majority will increasingly rely on photometric redshift estimates (see e.g. \citealt{Hogg98} and references contained therein) or clustering-based redshift estimates (see e.g. \citealt{Newman08} and references contained therein), enabling higher number density but noisier catalogs of galaxy positions.

Measurements of the cosmic microwave background (CMB) provide another window into the growth of large-scale structure, due to the lensing of the CMB photons as they free-stream through the Universe and are deflected (on the order of a few arcminutes) by the gravitational potentials of matter in their path. In the weak regime, gravitational lensing remaps the CMB temperature and polarization primary anisotropies in predictable ways that can be exploited to reconstruct high resolution maps of the projected matter density over the past 13 billion years (\citealt{ZaldarriagaSeljak99}, \citealt{HuOkamoto02}, \citealt{LewisChallinor06}). Detections of this mass lensing signal from the CMB have been made in a number of ways, including cross-correlations with other tracers of large-scale structure (see e.g.\ \citealt{PlanckI,Omori++19} for recent lists; also \citealt{Krolewski19}).

CMB lensing offers the advantage of directly probing the underlying distribution of dark matter, but suffers from information loss since it is a two-dimensional projection of the three-dimensional matter density integrated along the line of sight from the surface of last scattering $z \approx 1100$ to the present day. In contrast, galaxy samples with narrow redshift windows are relatively well localized in position but are biased tracers of dark matter due to the complex processes involved in galaxy formation. This leads to degeneracies between these galaxy bias parameters and cosmological parameters of interest such as $\sigma_8$ -- with recent surveys reporting a range of different inferences about the clustering amplitude \citep[e.g.][and references therein]{PlanckVI,Troxel18,Hikage19,Troster20,Philcox20,eBOSS20}. Cross-correlations between CMB lensing and galaxy catalogs thus provide a means to chart the growth of dark matter with time and break the degeneracy between galaxy physics and cosmology. Additionally, on a practical level, systematics in the galaxy sample are unlikely to be correlated to systematics in the CMB lensing maps, and a higher degree of uncertainty in the galaxy redshift distribution can also be tolerated due to the broad redshift kernel of the CMB lensing.

In this work, we leverage the high number density and completeness of the luminous red galaxy (LRG) target class as defined by DESI and selected from deep multi-band imaging, in combination with the all-sky CMB lensing convergence maps of the Planck collaboration \citep{PlanckVIII}, to detect a galaxy-matter cross-correlation at high significance out to small scales, $\ell_{\rm max} = 1000$. We jointly model the angular auto- and cross- spectra to probe the amplitude and evolution of the galaxy bias. In the absence of spectroscopic redshifts, we use a combination of photometric and clustering-based estimations of the galaxy redshift distribution. Within a simple linear bias model $P_{\rm gg}(k,z) \approx b_{\rm g}(z)^2 P_{\rm mm}(k,z)$, the advantage of the clustering-based method is that it allows us to measure an effective bias without assuming a bias evolution model. By comparing the results using photometric versus clustering redshift distributions, we also evaluate the impact of the uncertainty in the redshift distribution on the inferred parameters.

This paper is organized as follows: Section~\ref{sec:data} describes the lensing products and imaging data, and outlines the construction of the DESI-like LRG catalog. In Section~\ref{sec:dndz_pipe}, we characterize the redshift distribution of the galaxy sample based on angular cross-correlations with external spectroscopic catalogs, and present a framework for probing bias evolution using these results. Section~\ref{sec:master_pipe} outlines our methods for measuring and modelling angular power spectra and covariances on a partial sky. Section~\ref{sec:magbias} is devoted to determining and applying corrections for the effects of magnification bias. In Section~\ref{sec:results}, we present and model the resulting spectra, with Section~\ref{sec:results/halofit} fitting the linear Eulerian galaxy bias under the HaloFit \citep{Smith++03} prescription while Section~\ref{sec:results/lpt} interprets the results within a Lagrangian perturbation theory framework. Finally, in Section~\ref{sec:conclusions}, we summarize our findings and suggest future directions. 

Throughout, we work in co-moving coordinates and assume the fiducial cosmology of the Planck 2018 results (\citealt{PlanckVI}, Table 2, Column 7). All magnitudes are quoted as AB magnitudes, unless otherwise specified.