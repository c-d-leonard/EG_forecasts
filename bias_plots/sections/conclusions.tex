In this paper, we present a cross-correlation between DESI-like LRGs selected from DECaLS DR8 and all-sky CMB lensing maps from Planck, and report a detection significance of $S/N = 27.2$ over a wide range of scales from $\ell_{\rm min} = 30$ to $\ell_{\rm max} = 1000$. 

To correct for the effects of magnification bias on the galaxy-galaxy auto-spectrum and galaxy-convergence cross-spectrum, we calculate the slope of the LRG cumulative magnitude function at the limiting magnitude of the survey, determining a value of order unity, $s = 0.999 \pm 0.015$. We find that the resulting corrections to the spectra are on the order of 4-6\%. We also test the impact of tSZ bias in the lensing map, showing the associated errors on the galaxy-lensing cross-correlation to be highly sub-dominant to the overall lensing noise.

Within two different frameworks for modeling galaxy clustering and using two different methods for estimating the redshift distribution of the LRG sample, we fit the galaxy bias in multiple complementary ways and cross-check the results, both for internal consistency and to ascertain the impact of uncertainty in the redshift distribution on the inferred bias parameters.

\begin{enumerate}

    \item Under a simple linear bias times HaloFit model, using a photometric $\phi(z)$ and an assumed bias evolution $b(z) \propto D(z)^{-1}$, we determine best fit values for the present day bias $b_{\rm gg} = 1.64 \pm 0.02$ and $b_{\rm \kappa g} = 1.32 \pm 0.05$. This value of the galaxy bias is similar to the prediction in the DESI Final Design Report \citep{DESI16}, $b_{\rm LRG}(z) = 1.7/D(z)$, though $b_{\rm \kappa g}$ is lower by a statistically significant amount.  This could indicate either a failure of the model or input assumptions or that the fiducial ``Planck 2018'' cosmology is incorrect.
    
    \item Under a simple linear bias times HaloFit model, using a clustering $\phi(z)$ and an assumed bias evolution $b(z) \propto D(z)^{-1}$, we determine best fit values for the present day bias $b_{\rm gg} = 1.56 \pm 0.01$ and $b_{\rm \kappa g} = 1.32 \pm 0.05$. We note that the value of $b_{\rm gg}$ changes by $\sigma_{b_{\rm gg}} = 0.08$ in switching from the photometric estimate of $\phi(z)$ to the clustering estimate of $\phi(z)$, whereas the cross-correlation is far more robust to this uncertainty in the redshift distribution, with the inferred parameter $b_{\rm \kappa g}$ unchanged.
    
    \item Under a simple linear bias times HaloFit model, using a clustering $b(z)\phi(z)$ with bias evolution implicitly folded into the overall redshift kernel, we determine best fit values for the effective bias $b_{\rm eff} \approx b(z_{\rm eff} = 0.68)$, finding $b^{\rm eff}_{\rm gg} = 2.23 \pm 0.02$ and $b^{\rm eff}_{\rm \kappa g} = 1.88 \pm 0.07$. We find perfect consistency with the results of (ii) under the latter's assumed bias evolution.
    
    \item Using perturbation theory with a Lagrangian bias model, and using a photometric $\phi(z)$, we determine model parameters evaluated at the effective redshift $z_{\rm eff} = 0.68$. The Lagrangian bias parameter $b_1 = 1.31 \pm 0.05$, when converted into Eulerian bias $b = 1 + b_{1}$, agrees with the results of (iii) within the error found to be associated with uncertainty in the redshift distribution. Furthermore, after applying the bias evolution assumption $b(z) \propto D(z)^{-1}$, this result is also in perfect agreement with the results of (i).  In contrast to the HaloFit model, the perturbative model (with scale-dependent bias) provides a consistent, statistically good fit to both spectra over the full $\ell$-range considered with our fiducial cosmology.  However even with this model we find weak statistical preference for $C_\ell^{\rm \kappa g}$ to lie lower than the theoretical prediction.
    
\end{enumerate}

In summary, we find strong constraints on the present day and effective linear bias, with the largest errors on these inferred parameters originating from errors in the galaxy redshift distribution but having negligible effect on the cross bias term $b_{\rm \kappa g}$. We also present a united framework for modeling bias in a bias evolution agnostic way, and use this to validate the assumption of passive bias evolution for LRGs. In future works, we intend to use the same framework to perform joint constraints on cosmological and galaxy bias parameters.

As this work was nearing completion we became aware of a similar analysis by \citet{Hang20}.  Those authors computed the clustering of several photometric galaxy samples, constructed from the Legacy Survey data, and their cross-correlation with the Planck lensing maps.  Where our results overlap they are in agreement, with both analyses finding that the $\kappa g$ spectrum is lower than the predictions of a HaloFit-based model fit to the $gg$ auto-spectrum.